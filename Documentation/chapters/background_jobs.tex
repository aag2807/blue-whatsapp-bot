\section{Overview}
The Blue WhatsApp Bot system includes several background jobs that automate routine tasks. These jobs are implemented using Quartz.NET, a powerful job scheduling library for .NET applications.

\section{Job Configuration}
All background jobs are configured in the \texttt{CronJobExtensions.cs} file, which sets up the job scheduling and dependencies. The jobs are registered during application startup using the \texttt{ConfigureCronSchedulerJobs} extension method.

\section{Available Jobs}

\subsection{Delete Old Folders Job}
This job manages the system's log files by removing old log directories.

\begin{itemize}
    \item \textbf{Schedule:} Weekly on Monday at 00:00
    \item \textbf{Purpose:} Cleans up old log files to prevent disk space issues
    \item \textbf{Implementation:} \texttt{DeleteOldFoldersJob} class
\end{itemize}

\subsubsection{Implementation Details}
\begin{lstlisting}[language=CSharp]
public class DeleteOldFoldersJob : IJob
{
    private readonly IAppLogger _logger;

    public DeleteOldFoldersJob(IAppLogger logger)
    {
        _logger = logger;
    }

    Task IJob.Execute(IJobExecutionContext context)
    {
        string rootDirectory = Path.Combine(Directory.GetCurrentDirectory(), "logs");
        // ... implementation details ...
    }
}
\end{lstlisting}

\subsection{Reset Trip Capacities Job}
This job resets the capacity of all trips to their default value at the end of each day.

\begin{itemize}
    \item \textbf{Schedule:} Daily at 23:50
    \item \textbf{Purpose:} Ensures trip capacities are reset for the next day
    \item \textbf{Implementation:} \texttt{ResetTripCapacitiesJob} class
\end{itemize}

\subsubsection{Implementation Details}
\begin{lstlisting}[language=CSharp]
public class ResetTripCapacitiesJob : IJob
{
    private readonly IAppLogger _logger;
    private readonly ITripRepository _tripRepository;

    public ResetTripCapacitiesJob(IAppLogger logger, ITripRepository tripRepository)
    {
        _logger = logger;
        _tripRepository = tripRepository;
    }

    async Task IJob.Execute(IJobExecutionContext context)
    {
        try
        {
            _logger.LogInfo("Starting daily trip capacity reset job");
            var trips = await _tripRepository.GetAllTripsAsync();
            foreach (var trip in trips)
            {
                trip.MaxCapacity = 30;
                await _tripRepository.UpdateTripAsync(trip);
            }
            _logger.LogInfo($"Successfully reset capacities for {trips.Count()} trips");
        }
        catch (Exception ex)
        {
            _logger.LogError($"Failed to reset trip capacities. Error: {ex.Message}");
        }
    }
}
\end{lstlisting}

\section{Job Configuration}
The jobs are configured in the application startup using the following code:

\begin{lstlisting}[language=CSharp]
internal static void ConfigureCronSchedulerJobs(this WebApplicationBuilder builder)
{
    builder.Services.AddQuartz(q =>
    {
        q.UseMicrosoftDependencyInjectionJobFactory();

        // Configure DeleteOldFoldersJob
        var deleteFoldersJobKey = new JobKey("DeleteOldFoldersJob");
        q.AddJob<DeleteOldFoldersJob>(opts => opts.WithIdentity(deleteFoldersJobKey));

        q.AddTrigger(opts => opts
            .ForJob(deleteFoldersJobKey)
            .WithIdentity("DeleteOldFoldersTrigger")
            .WithSchedule(CronScheduleBuilder.WeeklyOnDayAndHourAndMinute(DayOfWeek.Monday, 0, 0))
        );

        // Configure ResetTripCapacitiesJob
        var resetCapacitiesJobKey = new JobKey("ResetTripCapacitiesJob");
        q.AddJob<ResetTripCapacitiesJob>(opts => opts.WithIdentity(resetCapacitiesJobKey));

        q.AddTrigger(opts => opts
            .ForJob(resetCapacitiesJobKey)
            .WithIdentity("ResetTripCapacitiesTrigger")
            .WithSchedule(CronScheduleBuilder.DailyAtHourAndMinute(23, 50))
        );
    });

    builder.Services.AddQuartzHostedService(q => q.WaitForJobsToComplete = true);
}
\end{lstlisting}

\section{Error Handling}
All background jobs include comprehensive error handling and logging:

\begin{itemize}
    \item Each job operation is wrapped in try-catch blocks
    \item Errors are logged using the \texttt{IAppLogger} interface
    \item Failed jobs are logged with detailed error messages
    \item The system continues to run even if a job fails
\end{itemize}

\section{Monitoring}
The system provides monitoring capabilities for background jobs:

\begin{itemize}
    \item Job execution status is logged
    \item Success/failure counts are tracked
    \item Detailed error messages are available in the logs
    \item Job execution times are recorded
\end{itemize}

\section{Adding New Jobs}
To add a new background job to the system:

\begin{enumerate}
    \item Create a new class implementing \texttt{IJob}
    \item Implement the \texttt{Execute} method
    \item Add the job configuration in \texttt{ConfigureCronSchedulerJobs}
    \item Register any required dependencies
    \item Configure the job schedule
\end{enumerate}

\section{Best Practices}
When working with background jobs in the system:

\begin{itemize}
    \item Always include proper error handling
    \item Log all important operations
    \item Use dependency injection for required services
    \item Keep job execution times reasonable
    \item Consider the impact on system resources
    \item Test jobs thoroughly before deployment
\end{itemize} 