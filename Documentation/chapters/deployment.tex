\section{Deployment Overview}
The Blue WhatsApp Bot is designed to be deployed in a production environment with high availability and scalability. This chapter outlines the deployment process, requirements, and best practices.

\section{System Requirements}

\subsection{Hardware Requirements}
\begin{itemize}
    \item \textbf{CPU}: Minimum 2 cores, recommended 4 cores
    \item \textbf{RAM}: Minimum 4GB, recommended 8GB
    \item \textbf{Storage}: Minimum 50GB SSD
    \item \textbf{Network}: Stable internet connection with minimum 10Mbps
\end{itemize}

\subsection{Software Requirements}
\begin{itemize}
    \item \textbf{Operating System}: Windows Server 2019 or later
    \item \textbf{.NET Runtime}: .NET 6.0 or later
    \item \textbf{Database}: SQL Server 2019 or later
    \item \textbf{IIS}: Version 10.0 or later
    \item \textbf{SSL Certificate}: Valid SSL certificate for HTTPS
\end{itemize}

\section{Deployment Process}

\subsection{Database Setup}
\begin{enumerate}
    \item Install SQL Server
    \item Create new database
    \item Run database migrations
    \item Configure database user and permissions
\end{enumerate}

\begin{lstlisting}[language=sql]
-- Create database
CREATE DATABASE BlueWhatsAppBot;

-- Create database user
CREATE LOGIN BlueWhatsAppBotUser WITH PASSWORD = 'StrongPassword123!';
USE BlueWhatsAppBot;
CREATE USER BlueWhatsAppBotUser FOR LOGIN BlueWhatsAppBotUser;

-- Grant permissions
GRANT SELECT, INSERT, UPDATE, DELETE ON SCHEMA::dbo TO BlueWhatsAppBotUser;
\end{lstlisting}

\subsection{Application Deployment}
\begin{enumerate}
    \item Publish application
    \item Configure IIS
    \item Set up application pool
    \item Configure SSL
    \item Set up logging
\end{enumerate}

\begin{lstlisting}[language=xml]
<!-- web.config -->
<?xml version="1.0" encoding="utf-8"?>
<configuration>
  <system.webServer>
    <handlers>
      <add name="aspNetCore" path="*" verb="*" modules="AspNetCoreModuleV2" resourceType="Unspecified" />
    </handlers>
    <aspNetCore processPath="dotnet" arguments=".\BlueWhatsapp.Api.dll" stdoutLogEnabled="true" stdoutLogFile=".\logs\stdout" />
  </system.webServer>
</configuration>
\end{lstlisting}

\section{Configuration}

\subsection{Environment Variables}
Required environment variables:

\begin{lstlisting}[language=json]
{
  "ConnectionStrings": {
    "DefaultConnection": "Server=localhost;Database=BlueWhatsAppBot;User Id=BlueWhatsAppBotUser;Password=StrongPassword123!;"
  },
  "WhatsAppApi": {
    "ApiKey": "your-api-key",
    "ApiUrl": "https://graph.facebook.com/v17.0/your-phone-number-id"
  },
  "Jwt": {
    "Key": "your-secret-key",
    "Issuer": "BlueWhatsAppBot",
    "Audience": "BlueWhatsAppBotUsers"
  },
  "Logging": {
    "LogDirectory": "C:\\Logs\\BlueWhatsAppBot"
  }
}
\end{lstlisting}

\subsection{IIS Configuration}
IIS application pool settings:

\begin{lstlisting}[language=xml]
<applicationPools>
  <add name="BlueWhatsAppBot" 
       autoStart="true" 
       managedRuntimeVersion="v4.0" 
       managedPipelineMode="Integrated">
    <processModel identityType="ApplicationPoolIdentity" />
  </add>
</applicationPools>
\end{lstlisting}

\section{Monitoring}

\subsection{Logging Configuration}
Configure logging in \texttt{appsettings.json}:

\begin{lstlisting}[language=json]
{
  "Logging": {
    "LogLevel": {
      "Default": "Information",
      "Microsoft": "Warning",
      "Microsoft.Hosting.Lifetime": "Information"
    },
    "File": {
      "Path": "C:\\Logs\\BlueWhatsAppBot\\app.log",
      "Append": true,
      "MinLevel": "Information",
      "FileSizeLimitBytes": 10485760,
      "MaxRollingFiles": 10
    }
  }
}
\end{lstlisting}

\subsection{Health Checks}
Configure health checks in \texttt{Startup.cs}:

\begin{lstlisting}[language=CSharp]
public void ConfigureServices(IServiceCollection services)
{
    services.AddHealthChecks()
        .AddSqlServer(Configuration["ConnectionStrings:DefaultConnection"])
        .AddCheck<WhatsAppApiHealthCheck>("WhatsApp API")
        .AddCheck<DiskStorageHealthCheck>("Disk Storage");
}

public void Configure(IApplicationBuilder app, IWebHostEnvironment env)
{
    app.UseHealthChecks("/health", new HealthCheckOptions
    {
        ResponseWriter = async (context, report) =>
        {
            context.Response.ContentType = "application/json";
            var response = new
            {
                status = report.Status.ToString(),
                checks = report.Entries.Select(x => new
                {
                    name = x.Key,
                    status = x.Value.Status.ToString(),
                    description = x.Value.Description
                })
            };
            await context.Response.WriteAsJsonAsync(response);
        }
    });
}
\end{lstlisting}

\section{Backup and Recovery}

\subsection{Database Backup}
Configure SQL Server backup jobs:

\begin{lstlisting}[language=sql]
-- Create backup job
USE msdb;
GO

EXEC dbo.sp_add_job
    @job_name = N'BlueWhatsAppBot_DailyBackup',
    @description = N'Daily backup of BlueWhatsAppBot database',
    @category_name = N'Database Maintenance',
    @owner_login_name = N'sa';
GO

EXEC dbo.sp_add_jobstep
    @job_name = N'BlueWhatsAppBot_DailyBackup',
    @step_name = N'Backup Database',
    @subsystem = N'TSQL',
    @command = N'BACKUP DATABASE BlueWhatsAppBot TO DISK = ''C:\Backups\BlueWhatsAppBot_$(ESCAPE_SQUOTE(DATE)).bak'' WITH INIT';
GO

EXEC dbo.sp_add_schedule
    @job_name = N'BlueWhatsAppBot_DailyBackup',
    @name = N'Daily Schedule',
    @freq_type = 4,
    @freq_interval = 1,
    @active_start_time = 010000;
GO
\end{lstlisting}

\subsection{Application Backup}
Configure application backup:

\begin{lstlisting}[language=powershell]
# Backup script
$backupPath = "C:\Backups\BlueWhatsAppBot"
$appPath = "C:\inetpub\wwwroot\BlueWhatsAppBot"
$date = Get-Date -Format "yyyy-MM-dd"

# Create backup directory
New-Item -ItemType Directory -Force -Path "$backupPath\$date"

# Backup application files
Copy-Item -Path "$appPath\*" -Destination "$backupPath\$date" -Recurse

# Backup logs
Copy-Item -Path "C:\Logs\BlueWhatsAppBot\*" -Destination "$backupPath\$date\Logs" -Recurse
\end{lstlisting}

\section{Security}

\subsection{SSL Configuration}
Configure SSL in IIS:

\begin{lstlisting}[language=xml]
<system.webServer>
  <security>
    <requestFiltering>
      <requestLimits maxAllowedContentLength="30000000" />
    </requestFiltering>
  </security>
  <rewrite>
    <rules>
      <rule name="HTTP to HTTPS" stopProcessing="true">
        <match url="(.*)" />
        <conditions>
          <add input="{HTTPS}" pattern="^OFF$" />
        </conditions>
        <action type="Redirect" url="https://{HTTP_HOST}/{R:1}" redirectType="Permanent" />
      </rule>
    </rules>
  </rewrite>
</system.webServer>
\end{lstlisting}

\subsection{Firewall Configuration}
Configure Windows Firewall:

\begin{lstlisting}[language=powershell]
# Allow HTTP traffic
New-NetFirewallRule -DisplayName "Allow HTTP" -Direction Inbound -Protocol TCP -LocalPort 80 -Action Allow

# Allow HTTPS traffic
New-NetFirewallRule -DisplayName "Allow HTTPS" -Direction Inbound -Protocol TCP -LocalPort 443 -Action Allow
\end{lstlisting}

\section{Maintenance}

\subsection{Log Rotation}
Configure log rotation in \texttt{appsettings.json}:

\begin{lstlisting}[language=json]
{
  "Serilog": {
    "WriteTo": [
      {
        "Name": "File",
        "Args": {
          "path": "C:\\Logs\\BlueWhatsAppBot\\log-.txt",
          "rollingInterval": "Day",
          "retainedFileCountLimit": 31
        }
      }
    ]
  }
}
\end{lstlisting}

\subsection{Database Maintenance}
Configure database maintenance jobs:

\begin{lstlisting}[language=sql]
-- Create maintenance job
USE msdb;
GO

EXEC dbo.sp_add_job
    @job_name = N'BlueWhatsAppBot_Maintenance',
    @description = N'Weekly database maintenance',
    @category_name = N'Database Maintenance',
    @owner_login_name = N'sa';
GO

-- Add maintenance steps
EXEC dbo.sp_add_jobstep
    @job_name = N'BlueWhatsAppBot_Maintenance',
    @step_name = N'Update Statistics',
    @subsystem = N'TSQL',
    @command = N'UPDATE STATISTICS BlueWhatsAppBot WITH FULLSCAN';
GO

EXEC dbo.sp_add_jobstep
    @job_name = N'BlueWhatsAppBot_Maintenance',
    @step_name = N'Rebuild Indexes',
    @subsystem = N'TSQL',
    @command = N'ALTER INDEX ALL ON BlueWhatsAppBot.dbo.Reservations REBUILD';
GO
\end{lstlisting}

\section{Best Practices}

\subsection{Deployment}
\begin{itemize}
    \item Use deployment scripts for consistency
    \item Implement blue-green deployment
    \item Test deployment in staging environment
    \item Maintain deployment documentation
    \item Use version control for configuration
\end{itemize}

\subsection{Monitoring}
\begin{itemize}
    \item Set up application monitoring
    \item Configure alert thresholds
    \item Monitor system resources
    \item Track API usage
    \item Monitor error rates
\end{itemize}

\subsection{Security}
\begin{itemize}
    \item Regular security updates
    \item SSL certificate renewal
    \item Access control review
    \item Security audit logging
    \item Regular vulnerability scanning
\end{itemize} 