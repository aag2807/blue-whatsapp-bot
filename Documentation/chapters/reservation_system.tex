\section{Reservation System Overview}
The Blue WhatsApp Bot's reservation system is designed to handle trip bookings through WhatsApp conversations, providing a seamless experience for users while maintaining accurate capacity management.

\section{Reservation Process}

\subsection{Conversation Flow}
The reservation process follows a structured conversation flow:

\begin{enumerate}
    \item \textbf{Welcome Message}: Initial greeting and introduction to the booking system
    \item \textbf{Date Selection}: User selects the desired trip date
    \item \textbf{Schedule Selection}: User chooses from available time slots
    \item \textbf{Hotel Selection}: User selects their pickup location
    \item \textbf{Confirmation}: System confirms the booking details
    \item \textbf{Completion}: Reservation is created and confirmation is sent
\end{enumerate}

\subsection{State Management}
The conversation state is managed through the \texttt{ConversationState} class:

\begin{lstlisting}[language=CSharp]
public class ConversationState
{
    public string UserNumber { get; set; }
    public string CurrentStep { get; set; }
    public Dictionary<string, string> UserData { get; set; }
    public DateTime LastInteraction { get; set; }
}
\end{lstlisting}

\section{Capacity Management}

\subsection{Trip Capacity}
Each trip has a maximum capacity that is managed through the \texttt{CoreTrip} model:

\begin{lstlisting}[language=CSharp]
public class CoreTrip
{
    public int Id { get; set; }
    public string TripName { get; set; }
    public int MaxCapacity { get; set; }
    public int CurrentReservations { get; set; }
    public int RemainingCapacity => MaxCapacity - CurrentReservations;
    // ... other properties
}
\end{lstlisting}

\subsection{Capacity Reset}
Trip capacities are automatically reset daily at 23:50 through the \texttt{ResetTripCapacitiesJob}:

\begin{lstlisting}[language=CSharp]
public class ResetTripCapacitiesJob : IJob
{
    private readonly ITripRepository _tripRepository;
    private readonly IAppLogger _logger;

    public async Task Execute(IJobExecutionContext context)
    {
        try
        {
            var trips = await _tripRepository.GetAllTripsAsync();
            foreach (var trip in trips)
            {
                trip.CurrentReservations = 0;
                await _tripRepository.UpdateTripAsync(trip);
            }
            _logger.LogInfo("Trip capacities reset successfully");
        }
        catch (Exception ex)
        {
            _logger.LogError($"Error resetting trip capacities: {ex.Message}");
            throw;
        }
    }
}
\end{lstlisting}

\section{Reservation Validation}

\subsection{Capacity Check}
Before creating a reservation, the system validates the trip capacity:

\begin{lstlisting}[language=CSharp]
public async Task<CoreReservation> CreateReservationAsync(CoreReservation reservation)
{
    var trip = await _tripRepository.GetTripByIdAsync(reservation.TripId);
    if (trip == null)
        throw new ReservationException("Trip not found");

    if (trip.CurrentReservations >= trip.MaxCapacity)
        throw new TripCapacityException("Trip is at full capacity");

    // ... create reservation logic
}
\end{lstlisting}

\subsection{Date Validation}
The system ensures reservations are made for valid dates:

\begin{lstlisting}[language=CSharp]
private bool IsValidReservationDate(DateTime date)
{
    return date.Date >= DateTime.Today && 
           date.Date <= DateTime.Today.AddDays(30);
}
\end{lstlisting}

\section{Real-time Updates}

\subsection{SignalR Integration}
Reservations are broadcast in real-time through SignalR:

\begin{lstlisting}[language=CSharp]
public async Task SaveReservation(CoreReservation reservation)
{
    var created = await _reservationRepository.CreateReservationAsync(reservation);
    await Clients.All.SendAsync("ReservationCreated", created);
}
\end{lstlisting}

\subsection{Capacity Updates}
Trip capacity changes are immediately reflected:

\begin{lstlisting}[language=CSharp]
public async Task UpdateTripCapacity(int tripId)
{
    var trip = await _tripRepository.GetTripByIdAsync(tripId);
    trip.CurrentReservations++;
    await _tripRepository.UpdateTripAsync(trip);
    await Clients.All.SendAsync("TripCapacityUpdated", trip);
}
\end{lstlisting}

\section{Error Handling}

\subsection{Reservation Exceptions}
Custom exceptions for reservation-related errors:

\begin{lstlisting}[language=CSharp]
public class ReservationException : Exception
{
    public ReservationException(string message) : base(message) { }
}

public class TripCapacityException : Exception
{
    public TripCapacityException(string message) : base(message) { }
}
\end{lstlisting}

\subsection{Error Recovery}
The system includes mechanisms for handling failed reservations:

\begin{lstlisting}[language=CSharp]
public async Task HandleFailedReservation(CoreReservation reservation, Exception ex)
{
    _logger.LogError($"Reservation failed: {ex.Message}");
    // Notify user of failure
    await _messageCreator.CreateErrorMessage(reservation.PhoneNumber);
    // Attempt to rollback any partial changes
    await RollbackReservationChanges(reservation);
}
\end{lstlisting}

\section{Reservation Queries}

\subsection{Common Queries}
The system provides various methods to query reservations:

\begin{lstlisting}[language=CSharp]
public interface IReservationRepository
{
    Task<IEnumerable<CoreReservation>> GetReservationsByDate(DateTime date);
    Task<IEnumerable<CoreReservation>> GetReservationsByTripId(int tripId);
    Task<IEnumerable<CoreReservation>> GetReservationsByPhoneNumber(string phoneNumber);
}
\end{lstlisting}

\subsection{Reporting}
Reservation data can be aggregated for reporting:

\begin{lstlisting}[language=CSharp]
public async Task<ReservationReport> GenerateDailyReport(DateTime date)
{
    var reservations = await _reservationRepository.GetReservationsByDate(date);
    return new ReservationReport
    {
        Date = date,
        TotalReservations = reservations.Count(),
        ReservationsByTrip = reservations.GroupBy(r => r.TripId)
            .ToDictionary(g => g.Key, g => g.Count())
    };
}
\end{lstlisting}

\section{Best Practices}

\subsection{Reservation Management}
\begin{itemize}
    \item Always validate capacity before creating reservations
    \item Use transactions for atomic operations
    \item Implement proper error handling and recovery
    \item Maintain audit trails of all changes
    \item Send confirmations for all successful operations
\end{itemize}

\subsection{Performance Considerations}
\begin{itemize}
    \item Use appropriate indexes for frequent queries
    \item Implement caching for static data
    \item Optimize database queries
    \item Monitor capacity thresholds
    \item Implement rate limiting for API endpoints
\end{itemize} 