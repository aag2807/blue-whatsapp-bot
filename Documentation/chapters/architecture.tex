\section{Architecture Overview}
The Blue WhatsApp Bot follows a clean architecture pattern, separating the system into distinct layers with clear responsibilities and dependencies. The architecture is designed to be maintainable, testable, and scalable.

\section{Architecture Layers}

\subsection{Core Layer}
The innermost layer containing the business logic and domain models.

\begin{itemize}
    \item \textbf{Models:} Core domain entities and value objects
    \item \textbf{Interfaces:} Repository and service contracts
    \item \textbf{Enums:} Domain-specific enumerations
    \item \textbf{Utils:} Utility classes and helpers
\end{itemize}

\subsection{Boundaries Layer}
The layer responsible for implementing the interfaces defined in the Core layer.

\begin{itemize}
    \item \textbf{Persistence:} Database context and configurations
    \item \textbf{Repositories:} Implementation of repository interfaces
    \item \textbf{Models:} Database entity models
    \item \textbf{Configurations:} Entity Framework configurations
\end{itemize}

\subsection{API Layer}
The outermost layer handling HTTP requests and external integrations.

\begin{itemize}
    \item \textbf{Controllers:} API endpoints and request handling
    \item \textbf{Hubs:} SignalR hubs for real-time communication
    \item \textbf{Extensions:} Application configuration and middleware
    \item \textbf{Views:} Backoffice interface views
\end{itemize}

\section{Design Patterns}

\subsection{Repository Pattern}
Used for data access abstraction:

\begin{lstlisting}[language=CSharp]
public interface IReservationRepository
{
    Task<IEnumerable<CoreReservation>> GetAllReservationsAsync();
    Task<CoreReservation?> GetReservationByIdAsync(int id);
    Task<CoreReservation> CreateReservationAsync(CoreReservation reservation);
    // ... other methods
}
\end{lstlisting}

\subsection{Dependency Injection}
Used throughout the application for loose coupling:

\begin{lstlisting}[language=CSharp]
public class Startup
{
    public void ConfigureServices(IServiceCollection services)
    {
        services.AddScoped<IReservationRepository, ReservationRepository>();
        services.AddScoped<ITripRepository, TripRepository>();
        // ... other registrations
    }
}
\end{lstlisting}

\subsection{SignalR Hub Pattern}
Used for real-time communication:

\begin{lstlisting}[language=CSharp]
public class ReservationsHub : Hub
{
    private readonly IReservationRepository _reservationRepository;
    
    public ReservationsHub(IReservationRepository reservationRepository)
    {
        _reservationRepository = reservationRepository;
    }
    
    // ... hub methods
}
\end{lstlisting}

\section{Data Flow}

\subsection{Reservation Process}
\begin{enumerate}
    \item WhatsApp message received
    \item Message processed by conversation handler
    \item Reservation created in database
    \item Real-time update sent to backoffice
    \item Confirmation sent to user
\end{enumerate}

\subsection{Capacity Management}
\begin{enumerate}
    \item Trip capacity checked before reservation
    \item Capacity updated in real-time
    \item Daily reset at 23:50
    \item Notifications sent when capacity changes
\end{enumerate}

\section{Cross-Cutting Concerns}

\subsection{Logging}
Implemented using a custom logging interface:

\begin{lstlisting}[language=CSharp]
public interface IAppLogger
{
    void LogInfo(string message);
    void LogError(string message);
    void LogWarning(string message);
}
\end{lstlisting}

\subsection{Error Handling}
Global error handling middleware:

\begin{lstlisting}[language=CSharp]
public class ErrorHandlingMiddleware
{
    private readonly RequestDelegate _next;
    private readonly IAppLogger _logger;

    public async Task InvokeAsync(HttpContext context)
    {
        try
        {
            await _next(context);
        }
        catch (Exception ex)
        {
            _logger.LogError($"Error: {ex.Message}");
            // ... error handling logic
        }
    }
}
\end{lstlisting}

\section{Security}

\subsection{Authentication}
\begin{itemize}
    \item JWT-based authentication
    \item Role-based authorization
    \item Secure password hashing
\end{itemize}

\subsection{Data Protection}
\begin{itemize}
    \item HTTPS enforcement
    \item Input validation
    \item SQL injection prevention
    \item XSS protection
\end{itemize}

\section{Performance Considerations}

\subsection{Caching}
\begin{itemize}
    \item In-memory caching for frequently accessed data
    \item Response caching for static content
    \item Distributed caching support
\end{itemize}

\subsection{Database Optimization}
\begin{itemize}
    \item Indexed queries
    \item Efficient joins
    \item Connection pooling
    \item Query optimization
\end{itemize}

\section{Scalability}

\subsection{Horizontal Scaling}
\begin{itemize}
    \item Stateless design
    \item Load balancing support
    \item Distributed caching
    \item Database sharding capability
\end{itemize}

\subsection{Vertical Scaling}
\begin{itemize}
    \item Resource optimization
    \item Connection pooling
    \item Memory management
    \item Thread pool configuration
\end{itemize} 