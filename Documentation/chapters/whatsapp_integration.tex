\section{WhatsApp Integration Overview}
The Blue WhatsApp Bot integrates with the WhatsApp Business API to handle automated conversations and manage reservations. The integration provides a seamless experience for users while maintaining robust error handling and logging.

\section{Message Handling}

\subsection{Message Types}
The system handles various types of WhatsApp messages:

\begin{lstlisting}[language=CSharp]
public enum MessageType
{
    Text,
    Image,
    Document,
    Location,
    Contact
}

public class WhatsAppMessage
{
    public string From { get; set; }
    public string To { get; set; }
    public MessageType Type { get; set; }
    public string Content { get; set; }
    public DateTime Timestamp { get; set; }
}
\end{lstlisting}

\subsection{Message Processing}
Messages are processed through a dedicated handler:

\begin{lstlisting}[language=CSharp]
public class WhatsAppMessageHandler
{
    private readonly IMessageCreator _messageCreator;
    private readonly IAppLogger _logger;
    private readonly Dictionary<string, ConversationState> _conversationStates;

    public async Task HandleMessage(WhatsAppMessage message)
    {
        try
        {
            var state = GetOrCreateConversationState(message.From);
            await ProcessMessage(message, state);
        }
        catch (Exception ex)
        {
            _logger.LogError($"Error processing message: {ex.Message}");
            await SendErrorMessage(message.From);
        }
    }

    private async Task ProcessMessage(WhatsAppMessage message, ConversationState state)
    {
        switch (state.CurrentStep)
        {
            case "Welcome":
                await HandleWelcomeStep(message, state);
                break;
            case "DateSelection":
                await HandleDateSelection(message, state);
                break;
            // ... other steps
        }
    }
}
\end{lstlisting}

\section{Conversation Flow}

\subsection{Step Management}
The conversation flow is managed through distinct steps:

\begin{lstlisting}[language=CSharp]
public class ConversationStep
{
    public string Name { get; set; }
    public string Message { get; set; }
    public string NextStep { get; set; }
    public Func<string, bool> Validation { get; set; }
}

public class ConversationManager
{
    private readonly Dictionary<string, ConversationStep> _steps;

    public ConversationManager()
    {
        _steps = new Dictionary<string, ConversationStep>
        {
            {
                "Welcome",
                new ConversationStep
                {
                    Name = "Welcome",
                    Message = "Welcome to Blue WhatsApp Bot! Let's book your trip.",
                    NextStep = "DateSelection",
                    Validation = _ => true
                }
            },
            // ... other steps
        };
    }
}
\end{lstlisting}

\subsection{Message Templates}
Predefined message templates for different scenarios:

\begin{lstlisting}[language=CSharp]
public interface IMessageCreator
{
    CoreMessageToSend CreateWelcomeMessage(string userNumber);
    CoreMessageToSend CreateDatePromptMessage(string userNumber);
    CoreMessageToSend CreateSchedulePromptMessage(string userNumber);
    CoreMessageToSend CreateHotelPromptMessage(string userNumber);
    CoreMessageToSend CreateConfirmationMessage(string userNumber, CoreReservation reservation);
    CoreMessageToSend CreateTripFullMessage(string userNumber);
    CoreMessageToSend CreateErrorMessage(string userNumber);
}
\end{lstlisting}

\section{API Integration}

\subsection{WhatsApp API Client}
Integration with WhatsApp Business API:

\begin{lstlisting}[language=CSharp]
public class WhatsAppApiClient
{
    private readonly HttpClient _httpClient;
    private readonly string _apiKey;
    private readonly string _apiUrl;

    public async Task SendMessage(string to, string message)
    {
        var request = new
        {
            messaging_product = "whatsapp",
            to = to,
            type = "text",
            text = new { body = message }
        };

        var response = await _httpClient.PostAsJsonAsync($"{_apiUrl}/messages", request);
        response.EnsureSuccessStatusCode();
    }

    public async Task SendTemplate(string to, string templateName, Dictionary<string, string> parameters)
    {
        var request = new
        {
            messaging_product = "whatsapp",
            to = to,
            type = "template",
            template = new
            {
                name = templateName,
                language = new { code = "en" },
                components = parameters.Select(p => new
                {
                    type = "body",
                    parameters = new[] { new { type = "text", text = p.Value } }
                })
            }
        };

        var response = await _httpClient.PostAsJsonAsync($"{_apiUrl}/messages", request);
        response.EnsureSuccessStatusCode();
    }
}
\end{lstlisting}

\section{Error Handling}

\subsection{API Errors}
Handling WhatsApp API errors:

\begin{lstlisting}[language=CSharp]
public class WhatsAppApiException : Exception
{
    public int StatusCode { get; }
    public string ErrorCode { get; }

    public WhatsAppApiException(string message, int statusCode, string errorCode)
        : base(message)
    {
        StatusCode = statusCode;
        ErrorCode = errorCode;
    }
}

public async Task HandleApiError(HttpResponseMessage response)
{
    var error = await response.Content.ReadFromJsonAsync<WhatsAppApiError>();
    throw new WhatsAppApiException(
        error.Message,
        (int)response.StatusCode,
        error.ErrorCode
    );
}
\end{lstlisting}

\subsection{Retry Logic}
Implementing retry mechanism for failed API calls:

\begin{lstlisting}[language=CSharp]
public class WhatsAppMessageSender
{
    private readonly IWhatsAppApiClient _apiClient;
    private readonly IAppLogger _logger;
    private readonly IAsyncPolicy<HttpResponseMessage> _retryPolicy;

    public WhatsAppMessageSender(IWhatsAppApiClient apiClient, IAppLogger logger)
    {
        _apiClient = apiClient;
        _logger = logger;
        _retryPolicy = Policy<HttpResponseMessage>
            .Handle<HttpRequestException>()
            .Or<WhatsAppApiException>()
            .WaitAndRetryAsync(3, retryAttempt =>
                TimeSpan.FromSeconds(Math.Pow(2, retryAttempt)));
    }

    public async Task SendMessageWithRetry(string to, string message)
    {
        await _retryPolicy.ExecuteAsync(async () =>
        {
            try
            {
                return await _apiClient.SendMessage(to, message);
            }
            catch (Exception ex)
            {
                _logger.LogError($"Error sending message: {ex.Message}");
                throw;
            }
        });
    }
}
\end{lstlisting}

\section{Message Templates}

\subsection{Template Management}
Managing WhatsApp message templates:

\begin{lstlisting}[language=CSharp]
public class TemplateManager
{
    private readonly IWhatsAppApiClient _apiClient;
    private readonly IAppLogger _logger;

    public async Task CreateTemplate(string name, string content, string category)
    {
        var request = new
        {
            name = name,
            language = "en",
            category = category,
            components = new[]
            {
                new
                {
                    type = "BODY",
                    text = content
                }
            }
        };

        await _apiClient.CreateTemplate(request);
    }

    public async Task<List<Template>> GetTemplates()
    {
        return await _apiClient.GetTemplates();
    }
}
\end{lstlisting}

\section{Best Practices}

\subsection{Message Handling}
\begin{itemize}
    \item Implement proper error handling for API calls
    \item Use retry mechanisms for transient failures
    \item Log all message interactions
    \item Validate message content before sending
    \item Implement rate limiting to prevent abuse
\end{itemize}

\subsection{Conversation Management}
\begin{itemize}
    \item Maintain conversation state
    \item Implement timeout handling
    \item Provide clear user instructions
    \item Handle unexpected user inputs
    \item Implement fallback options
\end{itemize}

\subsection{Security}
\begin{itemize}
    \item Secure API key storage
    \item Validate message sources
    \item Implement message encryption
    \item Monitor for suspicious activity
    \item Regular security audits
\end{itemize} 