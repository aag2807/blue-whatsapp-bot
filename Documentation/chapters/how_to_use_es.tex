\section{Descripción General de Uso}
Este capítulo proporciona una guía completa sobre cómo utilizar el sistema Blue WhatsApp Bot, desde la configuración inicial hasta la gestión de reservas y el monitoreo del sistema.

\section{Comenzando}

\subsection{Requisitos Previos}
Antes de usar el sistema, asegúrese de tener:
\begin{itemize}
    \item Una cuenta válida de WhatsApp Business API
    \item SQL Server 2019 o posterior instalado
    \item Runtime .NET 6.0 o posterior
    \item IIS 10.0 o posterior configurado
    \item Certificado SSL válido para HTTPS
\end{itemize}

\subsection{Configuración Inicial}
\begin{enumerate}
    \item Configure la cadena de conexión a la base de datos en \texttt{appsettings.json}
    \item Configure las credenciales de la API de WhatsApp
    \item Configure el directorio de registro
    \item Configure las claves de autenticación JWT
\end{enumerate}

\section{Gestión de Viajes}

\subsection{Creación de un Viaje}
Para crear un nuevo viaje:
\begin{enumerate}
    \item Acceda a la interfaz de gestión de viajes
    \item Haga clic en "Crear Nuevo Viaje"
    \item Complete la información requerida:
        \begin{itemize}
            \item Nombre del Viaje
            \item ID de Ruta
            \item Capacidad Máxima
        \end{itemize}
    \item Seleccione los horarios disponibles
    \item Guarde el viaje
\end{enumerate}

\subsection{Gestión de Horarios de Viaje}
Para gestionar los horarios de viaje:
\begin{enumerate}
    \item Seleccione el viaje de la lista
    \item Haga clic en "Gestionar Horarios"
    \item Agregue o elimine horarios disponibles
    \item Establezca el estado del horario (activo/inactivo)
\end{enumerate}

\section{Proceso de Reserva}

\subsection{Flujo de Conversación de WhatsApp}
El sistema sigue este flujo de conversación:
\begin{enumerate}
    \item El usuario inicia la conversación
    \item El sistema envía un mensaje de bienvenida
    \item El usuario selecciona la fecha
    \item El usuario selecciona el horario
    \item El usuario selecciona el hotel
    \item El sistema confirma la reserva
    \item Se crea la reserva
\end{enumerate}

\subsection{Ejemplo de Conversación}
\begin{verbatim}
Bot: ¡Bienvenido a Blue WhatsApp Bot! Vamos a reservar tu viaje.
Usuario: Hola
Bot: Por favor, selecciona una fecha para tu viaje (DD/MM/AAAA)
Usuario: 15/05/2024
Bot: Horarios disponibles:
     1. 08:00 AM
     2. 10:00 AM
     3. 02:00 PM
     Por favor, selecciona un horario (1-3)
Usuario: 2
Bot: Por favor, selecciona tu hotel de recogida:
     1. Hotel A
     2. Hotel B
     3. Hotel C
Usuario: 1
Bot: Confirmando tu reserva:
     Fecha: 15/05/2024
     Hora: 10:00 AM
     Hotel: Hotel A
     ¿Es correcto? (Sí/No)
Usuario: Sí
Bot: ¡Tu reserva ha sido confirmada!
     ID de Reserva: #12345
     ¡Gracias por elegir Blue WhatsApp Bot!
\end{verbatim}

\section{Interfaz de Administración}

\subsection{Visualización de Reservas}
Para ver las reservas:
\begin{enumerate}
    \item Acceda al panel de Reservas
    \item Use filtros para buscar por:
        \begin{itemize}
            \item Fecha
            \item Hotel
            \item Viaje
            \item Nombre del cliente
            \item Número de teléfono
        \end{itemize}
    \item Vea los detalles de la reserva
    \item Exporte los datos si es necesario
\end{enumerate}

\subsection{Gestión de Capacidad}
Para gestionar la capacidad de viajes:
\begin{enumerate}
    \item Acceda al panel de Viajes
    \item Seleccione el viaje
    \item Vea la capacidad actual
    \item Ajuste la capacidad máxima si es necesario
    \item Monitoree los espacios restantes
\end{enumerate}

\section{Monitoreo y Mantenimiento}

\subsection{Estado del Sistema}
Monitoree el estado del sistema a través de:
\begin{itemize}
    \item Punto final de verificación de estado (/health)
    \item Archivos de registro en el directorio configurado
    \item Métricas de rendimiento de la base de datos
    \item Estadísticas de uso de la API
\end{itemize}

\subsection{Respaldo y Recuperación}
Tareas de mantenimiento regulares:
\begin{enumerate}
    \item Respaldos diarios de la base de datos
    \item Rotación de archivos de registro
    \item Respaldos de archivos de la aplicación
    \item Renovación del certificado SSL
\end{enumerate}

\section{Solución de Problemas}

\subsection{Problemas Comunes}
\begin{itemize}
    \item \textbf{Problemas de Conexión con la API de WhatsApp}
        \begin{itemize}
            \item Verifique las credenciales de la API
            \item Compruebe la conexión a internet
            \item Verifique los límites de tasa de la API
        \end{itemize}
    \item \textbf{Problemas de Conexión con la Base de Datos}
        \begin{itemize}
            \item Verifique la cadena de conexión
            \item Compruebe el estado de SQL Server
            \item Verifique los permisos de usuario
        \end{itemize}
    \item \textbf{Problemas de Gestión de Capacidad}
        \begin{itemize}
            \item Verifique el estado del viaje
            \item Compruebe la disponibilidad del horario
            \item Revise los conteos de reservas
        \end{itemize}
\end{itemize}

\subsection{Mensajes de Error}
Mensajes de error comunes y soluciones:
\begin{itemize}
    \item "El viaje está al máximo de capacidad" - El viaje ha alcanzado el máximo de reservas
    \item "Formato de fecha inválido" - Use el formato DD/MM/AAAA
    \item "Horario no disponible" - El horario seleccionado no está activo
    \item "Hotel no encontrado" - El hotel seleccionado no está en el sistema
\end{itemize}

\section{Mejores Prácticas}

\subsection{Uso del Sistema}
\begin{itemize}
    \item Monitoreo regular del estado del sistema
    \item Verificación diaria de respaldos
    \item Planificación de capacidad
    \item Plantillas de comunicación con usuarios
    \item Revisión de registros de errores
\end{itemize}

\subsection{Optimización de Rendimiento}
\begin{itemize}
    \item Mantenimiento regular de la base de datos
    \item Limpieza de archivos de registro
    \item Gestión de caché
    \item Monitoreo de límites de tasa de la API
    \item Seguimiento del uso de recursos
\end{itemize} 